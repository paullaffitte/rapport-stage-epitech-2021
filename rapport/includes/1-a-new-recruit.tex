\chapter{A New Recruit}

% Une partie destinée à un nouvel arrivant dans la société qui va reprendre / poursuivre le projet dans lequel vous avez été impliqué. Il faut donc lui décrire le contexte de l’entreprise, du projet, l’architecture générale de celui-ci, le contexte et l’organisation de l’équipe de travail, le tout de façon synthétique. Il faudra également y apporter les précisions quant aux difficultés rencontrées dans le projet.

This part has for objective to onboard a new recruit in a project that I worked on during my internship which is \saniscsicsi, developped as an \glsdef{open-source} software on \glsdef{github}. This software allows to perform \glsdef{dynamic-volume-provisioning} on \glsdef{k8s} and \glsdef{csi}-compliant \glsdeff{\glspl}{orchestrator}.

\clearpage

\section{Introduction}

In order to onboard you on our project \saniscsicsi, we will review some information you should know about it and about the company \gls{enix} more generally.

We will first discuss the company context, followed by the project context and its general architecture and finally describe the context and organisation of the project work team.

\section{Company context}

\gls{enix} SAS is a company selling \glsdef{cloud}, \glsdef{devops} and \glsdef{k8s} services in a \glsdeff{\acrshort}{b2b} business model. The company has been created in the early 2000s in the LAN parties environment. Sébastien, Romain, Alexandre, Jérôme and Laurent, student then, started this company in response to needs in infrastructure and networking during those events.

Today, \gls{enix} counts 15 employees and a few external collaborators. We work for some big clients like \gls{xbto}, \gls{tdf}, \gls{maif}, \gls{ulule}, \gls{airbus} and others.

\section{Project context}

\section{General project architecture}
\subsection{Difficulties encountered}

\section{Work team}
\subsection{Context}
\subsection{Organization}

\clearpage
